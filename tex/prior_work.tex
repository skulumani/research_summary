% !TEX Root = ../presentation.tex

\section[Prior Research]{Research into spacecraft mission design}
\subsection[Reachability Sets]{Orbital transfers via Reachability Sets}

\begin{frame}{Orbital Transfers} % -----------------------------------%
    \begin{itemize}
        \item \Emph{Reachability set} allows for systematic transfer design
        \begin{itemize}
            \item Transfer design on lower dimensional \Poincare surface
            \item Simple method to incorporate effects of low-thrust 
            \item Avoids the issue of determining initial conditions
        \end{itemize}
    \pause
        \item Alleviates many issues with previous approaches
        \begin{itemize}
            \item Initial states chosen from from the reachable set
            \item Indirect optimal control vs. direct optimal control
            \item Reachability set gives bounds on motion
        \end{itemize}    
    \end{itemize}

  \note[itemize]{
    \item Reachability set avoids the need to pick initial conditions
    \item We compute on a lower dimensional surface
  }
\end{frame} %--------------------------------------%

\subsubsection[Dynamic Systems Theory Review]{Dynamic Systems}

\begin{frame}{\Poincare map}
\begin{itemize}
    \item Intersection of a periodic orbit with a lower dimensional subspace
    \pause
        \begin{itemize}
            \item  \Emph{\Poincare section} - discrete map between intersections
        \end{itemize}
        \pause
    \item Useful for investigating the stability and structure 
    \pause
    \item Define a \Poincare section \( \Sigma \) 
        \begin{itemize}
            \item Used for initial and target periodic orbits
            \item Subspace for the \Emph{reachability set}
        \end{itemize}
\end{itemize}

\begin{center}
    \begin{scaletikzpicturetowidth}{0.3\textwidth}
    \begin{tikzpicture}[tdplot_main_coords,
          poincare/.style={opacity=.2,very thick,fill=blue},
          orbit/.style={very thick,black},
          orbit hidden/.style={very thick,dashed},
          grid/.style={very thin,gray!50},
          axis/.style={->,blue,thick},scale=0.6]

        % draw periodic orbit
        \onslide<4->{
            % draw a periodic orbit
            \coordinate (center) at (0,0,2);
            \node[above left] (x0) at (0,2,2) {\(\vecbf{x}_0\)};
            \filldraw (0,2,2) circle (3pt);
        }

        \onslide<5->{
            \tdplotdrawarc[orbit hidden]{(center)}{2}{90}{190}{}{};
            \tdplotdrawarc[orbit,<-]{(center)}{2}{-170}{90}{}{};
        }

        \onslide<6->{
            % nodes for the poincare section
            \node[label=above:\(\Sigma\)] (upper_right) at (0,5,5) {};
            \node[] (upper_left) at (0,1,5) {};
            \node[] (lower_left) at (0,1,0) {};
            \node[] (lower_right) at (0,5,0) {};

            % draw poincare section
            \draw[poincare] (upper_right.center) -- (upper_left.center) -- (lower_left.center) -- (lower_right.center) -- (upper_right.center);
        }
        
        \onslide<7->{
            \node[below right] (x1) at (0,3,2) {\(\vecbf{x}_1\)};
            \filldraw (0,3,2) circle (3pt);
        }

        \onslide<8->{
            \tdplotdrawarc[orbit hidden]{(center)}{3}{90}{199}{}{};
            \tdplotdrawarc[orbit,<-]{(center)}{3}{-161}{90}{}{};
        }
    \end{tikzpicture}
    \end{scaletikzpicturetowidth}
\end{center}

\end{frame}

\begin{frame}{Reachability Set}

\begin{itemize}
    \item Set of states achievable from a given initial condition over fixed \( t_f \) s.t. maximum control constraint
    \[
        R( \vecbf{x}_0, \mathcal{U} , t_f) = \braces{ \vecbf{x}_f \subseteq \mathcal{X} | \exists \vecbf{u} \in \mathcal{U}, \vecbf{x}(t_f) = \vecbf{x}_f }
    \]
    \pause
    \item Directly derivable from optimal control
    \item Frequently used for safety planning, e.g. air traffic avoidance
    \pause
    \item Extend to the design of orbital transfers
\end{itemize}

\end{frame}

\begin{frame}{Reachability Set on \Poincare section} % -----------------------------------%

\begin{itemize}
    \item Generate the reachability set on a \Poincare section, \( \Sigma \)
    \item Control input is chosen to enlarge the reachable set
\end{itemize}

\begin{center}
    \begin{tikzpicture}[tdplot_main_coords,
      poincare/.style={opacity=.2,very thick,fill=blue},
      orbit/.style={very thick,black},
      orbit hidden/.style={very thick,dashed},
      grid/.style={very thin,gray},
      axis/.style={->,blue,thick},
      reachability/.style={thick,blue}]

    % draw axes
    % \draw[axis] (0,0,0) -- (5,0,0) node[anchor=west]{$x$};
 %    \draw[axis] (0,0,0) -- (0,5,0) node[anchor=west]{$y$};
 %    \draw[axis] (0,0,0) -- (0,0,5) node[anchor=west]{$z$};

    % nodes for the poincare section
    \node[] (upper_right) at (0,5,5) {};
    \node[] (upper_left) at (0,1,5) {};
    \node[] (lower_left) at (0,1,0) {};
    \node[] (lower_right) at (0,5,0) {};

    \onslide<2->{
        % draw poincare section
        \draw[poincare] (upper_right.center) -- (upper_left.center) -- (lower_left.center) -- (lower_right.center) -- (upper_right.center);
        \node[label=above:\(\Sigma\)] at (upper_right) {};
        }

    % draw a periodic orbit
    \coordinate (center) at (0,0,2);
    \node[] (x0) at (0,3,2) {};
    \onslide<3-5>{
        \node[label=left:\(\vecbf{x}_0\)] at (x0) {};
        \filldraw (x0) circle (3pt);
    }

    \onslide<4-5>{
        \tdplotdrawarc[orbit hidden]{(center)}{3}{90}{200}{}{};
        \tdplotdrawarc[orbit,<-]{(center)}{3}{-160}{90}{}{};
        }

    \coordinate (reach) at (0,4.5,2);
    \tdplotsetthetaplanecoords{90}

    % draw poincare angles
    \onslide<5->{
        \draw[tdplot_rotated_coords,grid] (x0) -- (reach);
        \draw[tdplot_rotated_coords,grid] (x0) -- ++(-45:1.5);
        \tdplotdrawarc[tdplot_rotated_coords,grid]{(x0)}{0.5}{-45}{90}{above}{\(\phi\)};
        }

    % draw terminal state on reachability set
    \onslide<6->{
        \node[tdplot_rotated_coords,label=above:\(\vecbf{x}_f\)] (xf) at ($ (x0)+(-45:1.5) $) {};
        \filldraw (xf) circle (3pt);
        \node[label=below:\(\vecbf{x}_n\)] at (x0) {};
        }
    
    \onslide<7->{
        \node[tdplot_rotated_coords,label=below:\(J\)] at (xf) {};
        }

    \onslide<8->{
        
        \tdplotdrawarc[tdplot_rotated_coords,reachability]{(x0)}{1.5}{0}{360}{}{};
    }
    
\end{tikzpicture}
\end{center}

\end{frame} %--------------------------------------%


\subsubsection[Circular Restricted Three Body Problem]{Three Body Problem}

\begin{frame}%--------------------------------------------%
\frametitle{Three Body Problem}
    \begin{itemize}
        \item Transfer from \( L_1 \) orbit to periodic orbit near the Moon
        \item Bounded control input and fixed time horizon
    \end{itemize}
    \visible<2>{
        \begin{center}
            \includegraphics[width=0.7\textwidth,height=0.7\textheight,keepaspectratio]{2015AAS/moon_orbit.pdf}
        \end{center}
        }
    
    \note[itemize]{
        \item Introduce problem
        \item Possible use as a communication array
        \item Might actually be a distant retrograde orbit
        }
\end{frame} %--------------------------------------------%

\begin{frame}%------------------------------------------------%
    \frametitle{Reachable Set Transfer}
    \begin{itemize}
        \item Approximate the reachable set on the \Poincare section
        \begin{itemize}
            \item Generate many optimal solutions
        \end{itemize}
        \item<3-> Intersection point used to generate a transfer
        \begin{itemize}
            \item Shorter time of flight than uncontrolled dynamics
        \end{itemize}
    \end{itemize}
    \begin{center}
        \only<1-2>{\includegraphics[width=0.5\textwidth,height=0.7\textheight,keepaspectratio]{2015AAS/reach_trajectory}~}
        \only<3->{\includegraphics[width=0.5\textwidth,height=0.7\textheight,keepaspectratio]{2015AAS/reach_transfer}~}
        \only<2-3>{\includegraphics[width=0.5\textwidth,height=0.7\textheight,keepaspectratio]{2015AAS/poincare_compare}} 
    \end{center}
    
    \note[itemize]{
        \item Compare to reachable set approach
        \item Shorter time of flight
        \item Multiple shooting to solve TPBVP
        Vary \( \theta\) to change direction on section
        Linear interpolation to determine intersection on Poincar\'e section
        }
\end{frame} %--------------------------------------------------%

\begin{frame}{Geostationary  transfer} %----------------------------------------------------%
    \begin{itemize}
           \item Tranfer for geostationary orbit to a periodic orbit about \( L_1\)
           \item Multiple iterations of reachable set required for transfer
    \end{itemize}

    \begin{center}
       \only<1>{
       \includegraphics[width=0.5\textwidth,height=0.7\textheight,keepaspectratio]{figures/2015ACTA/initial_final.pdf}~
       \includegraphics[width=0.5\textwidth,height=0.7\textheight,keepaspectratio]{figures/2015ACTA/geo_transfer_full.pdf} 
       }

       \only<2>{
       \includegraphics[width=0.5\textwidth,height=0.7\textheight,keepaspectratio]{figures/2015ACTA/geo_transfer_zoom.pdf}~
       \includegraphics[width=0.5\textwidth,height=0.7\textheight,keepaspectratio]{figures/2015ACTA/poincare.pdf}
       }
    \end{center}

\end{frame} %--------------------------------------------%

\subsubsection[Transfers about 4769 Castalia]{Asteroid transfers}

% Results from 2016 AAS
\begin{frame}{Transfer Objective} %-----------------------------%

\begin{itemize}
    \item Goal is to transfer between two equatorial periodic orbits
    \item Typical scenario during study of an asteroid
\end{itemize}

\begin{center}
    \includegraphics[width=0.5\textwidth,height=0.7\textheight,keepaspectratio]{figures/2016AAS/initial_transfer.pdf}~
    \includegraphics[width=0.5\textwidth,height=0.7\textheight,keepaspectratio]{figures/2016AAS/initial_transfer_3d.pdf}
\end{center}

\end{frame}%-----------------------------%

\begin{frame}{Simulation}

\begin{itemize}
    \item Generate the reachability set through discretization of \( \phi_i \)
    \item Visualize \( \Sigma \in \R^4 \) through the use of two 2-D sections
    \pause
    \item Control input allows for large deviation in velocity components
\end{itemize}

\begin{center}
    \includegraphics[width=0.5\textwidth,height=0.7\textheight,keepaspectratio]{figures/2016AAS/poincare_xvsxdot.pdf}~
    \includegraphics[width=0.5\textwidth,height=0.7\textheight,keepaspectratio]{figures/2016AAS/poincare_zvszdot.pdf}
\end{center}

\end{frame}

\begin{frame}{Transfer Simulation}
    \begin{itemize}
        \item Four iterations of the reachable state to meet the target set
        \item Final transfer is computed with a fixed terminal state constraint
    \end{itemize}

    \begin{center}
        \includegraphics[width=0.5\textwidth,height=0.7\textheight,keepaspectratio]{figures/2016AAS/trajectory.pdf}~
        \includegraphics[width=0.5\textwidth,height=0.7\textheight,keepaspectratio]{figures/2016AAS/trajectory_3d.pdf}
    \end{center}

\end{frame}

\begin{frame}{Complete transfer}
\begin{itemize}
    \item We can visualize the complete trajectory in both the body and inertial frames
\end{itemize}

\begin{center}
  \animategraphics[controls,autoplay,loop,width=0.5\textwidth,height=0.7\textheight,keepaspectratio]{30}{animation/2016AAS/body/IMG}{00001}{01499}~\hfill
  \animategraphics[controls,autoplay,loop,width=0.5\textwidth,height=0.7\textheight,keepaspectratio]{30}{animation/2016AAS/inertial/IMG}{00001}{01499}
\end{center}

\end{frame}

\subsection[Autonomous Spacecraft attitude control with obstacles]{Constrained Attitude Control}


% Results from 2016 ACC

\begin{frame}{Spacecraft Orientation} %-----------------------------%

\begin{itemize}

    \item \Emph{Attitude Representation}: rotation matrix from body to inertial frame
     \[\SO =  \{R\in\R^{3\times 3}\,|\, R^TR=I,\;\mathrm{det}[R]=1\} . \]
    \item Rigid body attitude dynamics:
    \begin{gather*}
        J\dot\Omega + \Omega\times J\Omega = u+W(R,\Omega)\Delta , \quad \dot R = R\hat\Omega .
    \end{gather*}

    \item Sensor and obstacles defined by unit vectors in \( \R^3 \) 
        \begin{itemize}
            \item Body fixed sensor: \( r \in \S^2\)
            \item Inertially fixed hazard: \( v \in \S^2 \)
        \end{itemize} 
    \vs
    \item Hard cone constraint: \( r^T R^T v \leq \cos \theta \)
    
\end{itemize}
\end{frame}   %-----------------------------%

\begin{frame}{Control Objective} %---------------------------------------%

    \begin{block}{Nonlinear Control Design}
        Design control input \( u \) that stabilizes system from initial attitude \( R_0 \) to desired attitude \( R_d \) while avoiding obstacles
    \end{block}
    \pause
    \vs
    \begin{itemize}
        \item Avoid drawbacks of other approaches 
        \begin{itemize}
            \item \Emph{Geometric control} - analysis is conducted directly on \( \SO \) 
            \item \Emph{Barrier function} - allows for arbitrary amount of constraints
            \item \Emph{Efficient } - real time feedback control
            \item \Emph{Stability} - Lyapunov analysis gives rigourous stability proof
            \item \Emph{Adaptive} - handles system uncertainties
        \end{itemize}
    \end{itemize}
\end{frame}

\begin{frame}{Configuration Error Function} %-----------------------------%
\only<1>{
\begin{itemize}
    \item Error function quantifies ``distance'' to desired attitude
    \begin{align*}
            \Psi(R, R_d) = A(R, R_d) B(R) .
    \end{align*}
    \vs
    \item Combination of attractive and repulsive terms   
\end{itemize}
\begin{gather*}
    A(R, R_d) = \frac{1}{2} \tr{G \left( I - R_d^T R\right)} . \\ \\
    B_i(R) = 1 - \frac{1}{\alpha_i} \ln \left( - \frac{ r^T R^T v_i - \cos \theta_i}{1 + \cos \theta_i}\right) .
\end{gather*}     
}

\only<2>{
    \begin{itemize}
        \item Attractive well at the desired attitude
    \end{itemize}
    \begin{align*}
        A(R,R_d) = \frac{1}{2} \tr{G \left( I - R_d^T R\right)} .
    \end{align*}
    \begin{center} 
        \includegraphics[height=0.6\textheight]{figures/2016ACC/attract_error.pdf}
    \end{center}
}

\only<3>{
    \begin{itemize}
        \item Define a barrier around obstacles
    \end{itemize}
    \begin{align*}
        B_i(R) = 1 - \frac{1}{\alpha_i} \ln \left( - \frac{ r^T R^T v_i - \cos \theta_i}{1 + \cos \theta_i}\right).
    \end{align*}
    \begin{center}
        \includegraphics[height=0.6\textheight]{figures/2016ACC/avoid_error.pdf}
    \end{center}

}

\only<4>{
    \begin{itemize}
        \item Configuration error: \( \Psi : \Q \times \Q \to \R \) with control chosen to follow slope of \( \Psi \) to minimum at \( R_d\)
    \end{itemize}
    \begin{align*}
        \Psi(R, R_d) = A(R,R_d) B(R) .
    \end{align*}
    \begin{center}
        \includegraphics[height=0.6\textheight]{figures/2016ACC/combined_error.pdf}
    \end{center}
}
\end{frame}   %-----------------------------%

\begin{frame}{Numerical Simulation} %-----------------------------%

\begin{itemize}
    \item Simulate a S/C completing a yaw rotation
    \item Single obstacle in the path of sensor
\end{itemize}

\begin{center}
    \animategraphics[controls,autoplay,loop,width=0.5\textwidth]{8}{animation/2016ACC/single_noavoid/single_noavoid-}{0}{99}~
    \animategraphics[controls,autoplay,loop,width=0.5\textwidth]{8}{animation/2016ACC/single_avoid/single_avoid-}{0}{99}
\end{center}

\end{frame}%-----------------------------%

\begin{frame}{Multiple obstacles}%-------------------------------------%

\begin{itemize}
    \item Easily handle multiple arbitrary constraints 
    \begin{align*}
        \Psi = A(R) \bracket{1 + \sum_i C_i(R)} \quad C_i = B(R) - 1
    \end{align*}
\end{itemize}

\begin{center}
    \animategraphics[controls,autoplay,loop,width=0.5\textwidth]{8}{animation/2016ACC/multiple_avoid/multiple_avoid-}{0}{99}
\end{center}

\end{frame}%---------------------------------------%

\subsubsection[Quadrotor attitude control]{Hardware Implementation}

\begin{frame}{Hexrotor Experiment} %-----------------------------%
\begin{itemize}
    \item Attached to spherical joint to allow only attitude dynamics
\end{itemize}
\begin{center}
    \href{https://youtu.be/dsmAbwQram4?t=20s}{\includegraphics[height=0.7\textheight]{figures/2016ACC/hexrotor}}
\end{center}
\end{frame}   %-----------------------------%
