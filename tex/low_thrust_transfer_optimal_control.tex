
\subsection[Formulating the optimal control problem]{Optimal Control Problem}

\begin{frame}{Optimal Control Problem}
\begin{itemize}
    \item Reachability defined as distance between controlled and uncontrolled states
        \[
            J = -\frac{1}{2} \left( \vecbf{x}(t_f) - \vecbf{x}_{n}(t_f)\right)^T 
            Q
            \left( \vecbf{x}(t_f) - \vecbf{x}_{n}(t_f)\right) 
        \]
    
    \pause
    \item Terminal constraints - \( \vecbf{m}_i(\vecbf{x}_f) = 0\) ensures \( \Sigma \) intersection
    \pause
    \item Control constraint used to emulate realistic system
        \[
            c(\vecbf{u}) = \vecbf{u}^T \vecbf{u} - u_m^2 \leq 0 
        \]
\end{itemize}

\note[itemize]{
    \item Formulate the reachability in terms of an optimal control problem
}
\end{frame}

\begin{frame}{Shooting Method}
\begin{itemize}
    \item Shooting method used to solve boundary value problem
    \item Convergence is difficult with with single shooting
    \item<6-> Multiple shooting sub-divides the total interval
\end{itemize}

\only<2-5>{
\begin{scaletikzpicturetowidth}{\textwidth}
    \begin{tikzpicture}[scale=\tikzscale]
        \onslide<2->{
            \node[label=west:\(\vecbf{x}\)] (x0m) at (0,5) {};
            \node[] (xnm) at (20,5) {};
    
            \node[label=below:\(t_0\)] (t0) at (0,0) {};
            \node[label=below:\(t_f\)] (tf) at (20,0) {};

            \draw [<->,thick] (tf.center) -- (t0.center) -- (x0m.center);

            % nodes for each subsegment
            \node[label=left:\(\vecbf{x}_0\)] (x0) at (0,1) {\pgfuseplotmark{*}};
            \node[label=below left:\(\vecbf{\lambda}_0\)] at (x0) {};
            \node[label=right:\(\vecbf{x}_f\)] (xnminus) at (20,3) {\pgfuseplotmark{*}};
            \node[label=below right:\(\vecbf{\lambda}_f\)] at (xnminus) {};
        }


        % shooting attempts
        \onslide<3->{
            \draw [->,dashed] (x0.center) to [bend left=20] ($ (xnminus.center)+(0,1) $);
        }
        \onslide<4->{
            \draw [->,dashed] (x0.center) to [bend right=10] ($ (xnminus.center)+(0,-1) $);
        }
        % final correct shooting path
        \onslide<5->{
            \draw [->] (x0.center) to [bend left=10] (xnminus.center);
        }
    \end{tikzpicture}
\end{scaletikzpicturetowidth}
}

\only<6->{
\begin{scaletikzpicturetowidth}{\textwidth}
        \begin{tikzpicture}[scale=\tikzscale]
    % \draw[help lines] (0,0) grid (20,5) ; %grid
    \node[label=west:\(\vecbf{x}\)] (x0m) at (0,5) {};
    \node[] (x1m) at (5,5) {};
    \node[] (x2m) at (10,5) {};
    \node[] (x3m) at (15,5) {};
    \node[] (xnm) at (20,5) {};

    \node[label=below:\(t_0\)] (t0) at (0,0) {};
    \node[label=below:\(t_n\)] (tn) at (20,0) {};

    % nodes for each subsegment
    \node[label=left:\(\vecbf{x}_0\)] (x0) at (0,1) {\pgfuseplotmark{*}};
    \node[label=below left:\(\vecbf{\lambda}_0\)] at (x0) {};

    \node[label=right:\(\vecbf{x}_n\)] (xnminus) at (20,3) {\pgfuseplotmark{*}};
    \node[label=below right:\(\vecbf{\lambda}_n\)] at (xnminus) {};

    % draw axes
    \draw [<->,thick] (tn.center) -- (t0.center) -- (x0m.center);


    \onslide<7->{
        
        \node[label=below:\(t_1\)] (t1) at (5,0) {};
        \node[label=below:\(t_2\)] (t2) at (10,0) {};
        \node[label=below:\(t_{n-1}\)] (t3) at (15,0) {};
        
                \node[label=right:\(\vecbf{x}_1^{-}\)] (x1minus) at (5,2) {\pgfuseplotmark{*}};
        \node[label={135:\(\vecbf{x}_1^{+}\)}] (x1plus) at (5,3) {\pgfuseplotmark{o}};

                \node[label=right:\(\vecbf{x}_2^{-}\)] (x2minus) at (10,2) {\pgfuseplotmark{*}};
        \node[label={135:\(\vecbf{x}_2^{+}\)}] (x2plus) at (10,3) {\pgfuseplotmark{o}};

                \node[label=right:\(\vecbf{x}_{n-1}^{-}\)] (x3minus) at (15,2) {\pgfuseplotmark{*}};
        \node[label={135:\(\vecbf{x}_{n-1}^{+}\)}] (x3plus) at (15,3) {\pgfuseplotmark{o}};
        % draw segement dividers
        \draw [dashed,thick] (t1.center) -- (x1m.center);
        \draw [dashed,thick] (t2.center) -- (x2m.center);
        \draw [dashed,thick] (t3.center) -- (x3m.center);
        \draw [dashed,thick] (tn.center) -- (xnm.center);
    }

    \onslide<8->{

        \draw [->] (x0.center) to [bend left=30] (x1minus.center);
    }

    \onslide<9->{

        \draw [->] (x1plus.center) to [bend left=30] (x2minus.center);
    }

    \onslide<10->{
        % dotted lines here

        \draw[decorate sep={2mm}{10mm},fill] (x2plus.east) to [bend left=30] (x3minus.west);
    }

    \onslide<11->{

        \draw [->] (x3plus.center) to [bend left=30] (xnminus.center);
    }


\end{tikzpicture}
\end{scaletikzpicturetowidth}
}
\note[itemize]{
    \item The euler lagrange equations transforms from an optimal control problem to a two point boundary value problem
    \item in the shooting method we try to choose initial conditions \( \vecbf{\lambda}_0\) to satisfy some set of terminal conditions
    \item Again it's prone to sensitivty so we instead use a multiple shooting method 
    \item We divide the time interval into smaller sub segments and add additional interior constraints to ensure a continuous trajectory
    \item Each sub stage is then a smaller single shooting approach
    \item this ends up reducing the sensitivity of the terminal states to variations of the intial conditions
}
\end{frame}

\begin{frame}{Solving the Optimal Control Problem}

\begin{itemize}
    \item Shooting method to solve the necessary conditions
    \pause
    \item Approximate the reachable set via \( \phi_i \) 
    \begin{itemize}
        \item Parameterize a direction on \( \Sigma \)
    \end{itemize}
    \pause
    \item From the reachable set we chose the state which minimizes \( d \) 
    \item Compute another reachable set if target is not feasible
\end{itemize}
\[
    d = \norm{\vecbf{x}_f - \vecbf{x}_t} 
\]

\end{frame}

