% !TEX Root = ../proposal.tex

\section[PhD Research]{PhD George Washington University}

\subsection[Overview]{Overview}

\begin{frame}{Overview of research}
    \begin{itemize}
        \item Reachability
            \begin{itemize}
                \item Three Body Problem
                \item 3D transfers around asteroids
            \end{itemize}
        \item Constrained Attitude Control
        \item Asteroid Operations
            \begin{itemize}
                \item Coupled dynamics
                \item Geometric Control for Landing
                \item SLAM
            \end{itemize}
    \end{itemize}
\end{frame}

\begin{frame}{Asteroid Missions}
\begin{itemize}
    \item Science - insight into the early formation of the solar system
    \item Mining - vast quantities of useful materials
    \item Impact - high risk from hazardous near-Earth asteroids
\end{itemize}    

\begin{center}
    \includegraphics[height=0.3\textheight,width=0.5\textwidth,keepaspectratio]{figures/2016AAS/near_mos_20001203_full.jpg}
    ~
    \includegraphics[height=0.3\textheight,width=0.5\textwidth,keepaspectratio]{figures/2016AAS/Itokawa8_hayabusa_1210.jpg}
\end{center}
\end{frame}

\begin{frame}{Asteroid Mining}
    \begin{itemize}
      \item Useful materials can be extracted from asteroids to support:
      \begin{itemize}
          \item Propulsion, construction, life support, agriculture, and precious/strategic metals
      \end{itemize}
      \item Commercialization of near-Earth asteroids is feasible
    \end{itemize}

\pause

\begin{center}
\small
    \begin{tabular}{|l|r|r|}
        \hline 
        Element & Price (\SI{}{\$\per\kilo\gram}) & Sales (\SI{}{\$M\per\year}) \\
        \hline \hline 
        Phosphorous (P) & \num{0.08}  & \num{2167} \\
        Gallium (Ga) & \num{300.00}  & \num{1544} \\
        Germanium (Ge) & \num{745.00} & \num{6145} \\
        \hline \hline 
        Platinum (Pt) & \num{12394.00} & \num{1705} \\
        Gold (Au) & \num{12346.00} & \num{49} \\
        Osmium (Os) & \num{12860.00} & \num{307} \\
        \hline
    \end{tabular}
\end{center}

\end{frame}

\subsection[Electric Propulsion]{Electric Propulsion}  

\begin{frame} \label{slide:lowthrust_vehicles}%-----------------------------%
\frametitle{Low-thrust vehicles} % electric propulsion
\begin{itemize}
    \item Low-thrust orbital transfers offer increased mission oportunities
    \begin{itemize}
        \item Electric propulsion is increasing in capability
        \item Offers much higher specific impulse than chemical engines 
        \item Requires much longer operating periods for maneuvers 
        \item Enables long duration missions with frequent thrusting
    \end{itemize}
\end{itemize}

\begin{center}
    \includegraphics[height=0.4\textheight,width=0.5\textwidth,keepaspectratio]{figures/2016AAS/patriot_plume.jpg}
    ~
    \includegraphics[height=0.4\textheight,width=0.5\textwidth,keepaspectratio]{figures/2016AAS/deepspace1.jpg}
\end{center}
\hyperlink{slide:propulsion}{\beamergotobutton{Ideal Rockets}}
\end{frame}   %-----------------------------%

\subsection[Spacecraft Autonomy]{Spacecraft Autonomy}
% why study the coupled attitude/translational problem

\begin{frame}[t]{Spacecraft Autonomy} %-----------------------------%
\begin{itemize}
    \item Autonomous control of space vehicles is critical
    \begin{itemize}
        \item Avoid extensive planning and interaction by operators
        \item Ability to operate safely with system uncertainty 
        \item Independently navigate hazards and handle possible failures
    \end{itemize}
\end{itemize}
\visible<2>{
\begin{center}
    \includegraphics[width=0.5\textwidth,height=0.35\textheight,keepaspectratio]{figures/2016ACC/hubble.jpg}\hfill
    \includegraphics[width=0.5\textwidth,height=0.4\textheight,keepaspectratio]{figures/osires_rex.png}
\end{center}
}
\note[itemize]{
    \item Autonomy is a key component to enable asteroid missions
}
\end{frame}   %-----------------------------%
